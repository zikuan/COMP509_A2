\documentclass[12pt]{article}
\usepackage{amsmath, amssymb}
\usepackage{MnSymbol}
\usepackage{listings}
\usepackage{graphicx}
\graphicspath{{./}}
\renewcommand{\phi}{\varphi}

\title{COMP 409: Assignment 2}

\author{Partner 1 and Partner 2}
%Replace Partner 1 and 2 with your names.

\begin{document}
\maketitle


\subsubsection*{Hours Worked} 
%Replace Partner 1 and 2 with your names and replace X, Y and Z with the corresponding number of hours you spent on the homework working alone and together.

\noindent\emph{Jointly}: 5 hours

\noindent\emph{Yushu Liu}: 6 hours

\noindent\emph{Zikuan Zhang}: 6 hours

\newpage
\section*{Propositional Semantics}
%Write your solution to problem 1 here. 
\begin{center}
	\includegraphics[width=\textwidth]{./semantics.png}
\end{center}
\begin{lstlisting}[language=bash]
>stack ghci semantics.hs

*Main> evaluate psi i1
True
*Main> evaluate psi i2
True
\end{lstlisting}

\newpage
\section*{Validity}
%Write your solution to problem 2 here.
\subsection*{Problem 1}
\subsubsection*{(a)}
Peirce's Law: 
\begin{align*}
    models((p \rightarrow q) \rightarrow p)) 
    & = (2^{prop} - models(p \rightarrow q) )\cup models(p)\\
    & = ((2^{prop} - (2^{prop} - models(p))\cup models(q))\cup models(p)\\
    & = ((2^{prop} - (2^{prop} - models(p)))\cap (2^{prop}-models(q)))\cup models(p)\\
    & =(models(p) \cap (2^{prop} - models(q)) \cup models(p)\\
\end{align*}
So we have: 
\begin{align*}
     models((p \rightarrow q) \rightarrow p)) \subseteq models(p)\\
     models((p \rightarrow q) \rightarrow p)) \models p\\
\end{align*}
Then $(((p \rightarrow q) \rightarrow p))\rightarrow p)$ is valid (by lemma  of logical implication and material implication).
\subsubsection*{(b)}
Ex Falso quolibet:
\begin{align*}
    models(false \rightarrow p) & = (2^{prop} - models(false)) \cup models(p)\\
    & =2^{prop} \cup models(p) \\
    & = 2^{prop}
\end{align*}
So the given formula is valid. 
\subsubsection*{(c)}
Excluded Middle:
\begin{align*}
    models(\neg p \vee p) &= models((\neg p)) \cup models(p) \\
    & = (2^{prop} - models(p)) \cup modles(p)\\
    & = 2^{prop}
\end{align*}
So the given formula is valid. 
\subsubsection*{(d)}
Weak Excluded Middle:
\begin{align*}
    models( (p \rightarrow q) \vee p) &= models(p \rightarrow q) \cup models(p) \\
    & = (2^{prop} - models(p)) \cup models(q) \cup models (p)\\
    & = 2^{prop}
\end{align*}
So the given formula is valid.
\subsection*{Problem 2}
We can prove the following statement: \\
If $\nvDash \alpha$ and $\nvDash \beta$ and $AP(\alpha) \cap AP(\beta) = \emptyset$, then $\nvDash (\alpha \vee \beta)$.\\
which is equivalent to the original statement since it is its contrapositive.\\[5pt]
If $\nvDash \alpha$ and $\nvDash \beta$, by definition there must exist at least one $\tau_{\alpha}$ and $\tau_{\beta}$ satisfying the following condition: \\
(1): $\tau_{\alpha} \in 2^{AP(\alpha)}$ and  $\tau_{\beta} \in 2^{AP(\beta)}$ \\
(2): $\tau_{\alpha} \nvDash \alpha$ and $\tau_{\beta} \nvDash \beta$\\
Let $\tau = \tau_{\alpha} \cup \tau_{\beta}$. Since $AP(\alpha) \cap AP(\beta) = \emptyset$, then in $\tau \in 2^{prop}$, the assignment of entirely $AP(\alpha)$ would be according to $\tau_{alpha}$. Similarly the assignment of entirely $AP(\beta)$ would be according to $\tau_{beta}$.\\
Then it must be the case that $\tau \nvDash \alpha$ and $\tau \nvDash \beta$ according to relevance lemma, therefore $\tau \nvDash (\alpha \vee \beta)$, and $\nvDash (\alpha \vee \beta)$.
Thus proved. 
\newpage
\section*{Logic Implication}
%Write your solution to problem 3 here. 
\subsection*{Problem 1}
$\wedge$\textbf{ is commutative and associative}:\\
Since set intersection is commutative, we have the following:\\[5pt]
\begin{align*}
models(\theta \wedge \phi)
&= models(\theta) \cap models(\phi) \\
&= models(\phi) \cap models(\theta) \\
&= models(\phi \wedge \theta)
\end{align*}
Similarly, since set intersection is associative: 
\begin{align*}
models((\theta \wedge \phi)\wedge \gamma)
&= models(\theta\wedge \phi) \cap models(\gamma) \\
&= models(\theta) \cap models(\phi) \cap models(\gamma) \\
&= models(\theta) \cap models(\phi \wedge \gamma) \\
&= models(\theta \wedge (\phi \wedge \gamma))\\
\end{align*}
Thus proved. \\[10pt]
$\vee$\textbf{ is commutative and associative}:\\
Since set union is commutative, we have the following:\\[5pt]
\begin{align*}
models(\theta \vee \phi)
&= models(\theta) \cup models(\phi) \\
&= models(\phi) \cup models(\theta) \\
&= models(\phi \vee \theta)
\end{align*}
Similarly, since set union is associative: 
\begin{align*}
models((\theta \vee \phi)\vee \gamma)
&= models(\theta\vee\phi) \cup models(\gamma) \\
&= models(\theta) \cup models(\phi) \cup models(\gamma) \\
&= models(\theta) \cup models(\phi \vee \gamma) \\
&= models(\theta \vee (\phi \vee \gamma))\\
\end{align*}
Thus proved.\\[10pt]
$\rightarrow$ \textbf{is not commutative or associative:} \\
Define a world with following mapping: p = 1, q = 0. Then we have: 
\begin{align*}
    p \rightarrow q = 0\\
    q \rightarrow p = 1\\
\end{align*}
Since the two statement above is not logically equivalent , $\rightarrow$ is not commutative.\\[3pt]
Similarly define a world with following mapping: p = 0, q = 0, r = 0. Then we have: 
\begin{align*}
    ((p \rightarrow q) \rightarrow r)= 0\\
    (p \rightarrow(q \rightarrow r)) = 1\\
\end{align*}
Since the two statement above is not logically equivalent , $\rightarrow$ is not associative.\\[20pt]
$\longleftrightarrow$ \textbf{is commutative and associative:} \\
Since set intersection and set union is commutative, we have the following:\\[5pt]
\begin{align*}
models(\theta \leftrightarrow \phi)
&= ((2^{prop} - models(\theta) \cap(2^{prop} - models(\phi))) \cup (models(\theta) \cap models(\phi))\\
& = ((2^{prop} - models(\phi) \cap(2^{prop} - models(\theta))) \cup (models(\phi) \cap models(\theta))\\
& = models(\phi \leftrightarrow \theta)
\end{align*}
Similarly, using the regrouping rule from set theory: 
\begin{align*}
&models((\theta \leftrightarrow \phi)\leftrightarrow \gamma \\
&=models( \theta \leftrightarrow \phi) \cap models(\gamma)) \cup ((2^{prop} - models (\theta \leftrightarrow \phi)) \cap (2^{prop} - models(\gamma)))\\
\end{align*}
where,
\begin{align*}
models(\theta \leftrightarrow \phi) = ((2^{prop} - models(\theta) \cap(2^{prop} - models(\phi))) \cup (models(\theta) \cap models(\phi))\\
\end{align*}
We have: 
\begin{align*}
&models((\theta \leftrightarrow \phi)\leftrightarrow \gamma) \\
& = (models( \theta \leftrightarrow \phi) \cap models(\gamma)) \cup ((2^{prop} - models (\theta \leftrightarrow \phi)) \cap (2^{prop} - models(\gamma)))\\
& = (models( \theta) \cap models(\phi \leftrightarrow \gamma)) \cup ((2^{prop} - models( \theta)) \cap (2^{prop} - models((\phi \leftrightarrow \gamma)))\\
& = models(\theta \leftrightarrow (\phi \leftrightarrow \gamma))
\end{align*}
Thus proved. 
\subsection*{Problem 2}
De Morgan's Law:\\ 
By relevance lemma, for a propsition $\theta$ to be valid, we only need to show $models(\theta) = 2^{AP(\theta)}$. Since in the case of De Mogan's Law, $2^{AP(\theta)}$(for both statements) includes only 4 cases, we can list them as follow:

\begin{center}
\begin{tabular}{|l|l|l|l|l|}
\hline
       & p & q & $((\neg((p \wedge q)) \leftrightarrow ((\neg p) \vee (\neg q))$ & $((\neg (p \vee q)) \leftrightarrow ((\neg p) \wedge (\neg q))$ \\ \hline
Case 1 & 0 & 0 & 1         & 1        \\ \hline
Case 2 & 0 & 1 & 1         & 1        \\ \hline
Case 3 & 1 & 0 & 1         & 1        \\ \hline
Case 4 & 0 & 0 & 1         & 1        \\ \hline
\end{tabular}
\end{center}
Thus proved. 
\subsection*{Problem 3}
By definition, $\phi \rightmodels\leftmodels  \psi$ is equivalent to $\phi \models \psi$ and $\psi \models \phi$.\\
By the lemma the statement above is also equivalent to $\models (\phi \rightarrow \psi)$ and $\models (\psi \rightarrow \phi)$, which by definition, is equivalent to $\models (\phi \leftrightarrow \psi)$
\subsection*{Problem 4}
We have the following logical equivalency: $(\phi \wedge \neg \psi)$ is not satisfiable if and only if $\neg(\phi \wedge \neg \psi)$ is valid. (by definition of satisfiable and valid).\\
We then convert $\neg(\phi \wedge \neg \psi)$to $(\neg \phi \vee \psi)$ using de Morgan's Law.\\
$(\neg \phi \vee \psi)$ is valid is logically equivalent to $\models(\neg \phi \vee \psi)$. By the definition of $\rightarrow$, we can rewrite $\models(\neg \phi \vee \psi)$ as $\models (\phi \rightarrow \psi)$. Then by the lemma of material and logical implication, $\models (\phi \rightarrow \psi) \Leftrightarrow \phi \models \psi$.\\[5pt]
Thus proved. 
\subsection*{Problem 5}
For logical implication, let $\Phi, \Psi, \Theta \in 2^{FORM}$, we then have:\\[5pt]
$\bullet$ If $models(\Phi) \subseteq models(\Phi)$, by definition $\Phi \models \Phi$. So it is reflexive.\\[3pt]
$\bullet$ Let $\Phi \models \Psi$ and $\Psi \models \Theta$, then $models(\Phi) \subseteq models(\Psi)$ and $models(\Psi) \subseteq models(\Theta)$. We can use the fact that $\subseteq$ is transitive to get $models(\Phi) \subseteq models(\Theta)$, therefore $\Phi \models \Theta$. So it is transitive.\\[3pt]
$\bullet$ Let $\Phi = (p \vee q)$ and $\Psi = (p \wedge q)$. Then $models(\Phi) = {(p,q)}$ and $models(\Phi) = {{p}, {q}, {p, q}}$. In this case, we have $models(\Phi) \subseteq models(\Psi)$ but $models(\Psi) \nsubseteq models(\Phi)$, therefore $\Phi \models \Psi$ and $\Psi \nvDash \Phi$. So it is not symmetric. \\[10pt]
For logical equivalence, similarly:\\[5pt]
$\bullet$ If $models(\Phi) = models(\Phi)$, by definition $\Phi \vDash \leftmodels \Phi$. So it is reflexive.\\[3pt]
$\bullet$ Let $\Phi \models \leftmodels \Psi$ and $\Psi \models \leftmodels \Theta$, then $models(\Phi) = models(\Psi)$ and $models(\Psi) = models(\Theta)$. We can use the fact that $ = $ is transitive to get $models(\Phi) = models(\Theta)$, therefore $\Phi \models \Theta$. So it is transitive.\\[3pt]
$\bullet$ Let $\Phi \models \leftmodels \Psi$, then $models(\Phi) = models(\Psi)$, $\Psi \models \leftmodels \Phi$. So it is symmetric.
\subsection*{Problem 6}
By definition of logical equivalency, since $\phi \models \leftmodels \psi$, $models(\phi) = models(\psi)$. By definition of validity, $models(\phi) = 2^{prop} = models(\psi)$, so $\psi$ is also valid. 
\subsection*{Problem 7}
Let the first sentence be $\phi$ and second sentence be $\psi$.\\[5pt]
We have $models(\phi) = \{\{p\}, \{p,q,r\}, \{q\}, \{r\}\}$, and $models(psi) = \{\{p, q, r\}, \{\emptyset\}\}$. Since $models(\phi) \nsubseteq models(\psi)$ and $models(\psi) \nsubseteq models(\phi)$, neither sentence implies the other. 
\newpage
\subsection*{Problem 8}
Proof:\\[10pt]
By definition, we have $\Sigma \cup \{\alpha\} \models B$\\
\begin{align*}
    &\text{Iff } models(\Sigma \cup \{\alpha\}) \subseteq models(\beta)\\
    &\text{Iff } models(\Sigma \cup \{\alpha\}) \subseteq models(\beta)\\
    &\text{Iff } (models(\Sigma) \cap models(\alpha)) \cup (2^{prop} - models(\alpha)) \subseteq models(\beta) \cup (2^{prop} - models(\alpha))\\
\end{align*}
$\longrightarrow$:\\
For arbitrary $\sigma \in models(\Sigma)$, we then try to prove it is always true that $\sigma \in (models(\Sigma) \cap models(\alpha)) \cup (2^{prop} - models(\alpha))$.It must either be the case of $\sigma \in models(\alpha)$ or $\sigma \notin models(\alpha)$. In first case $\sigma \in (models(\Sigma) \cap models(\alpha))$, and in second case $\sigma \in (2^{prop} - models(\alpha))$ by definition.\\[3pt]
Following the partial proof above, using the transitivity of $\subseteq$, we have:
\begin{align*}
    &(models(\Sigma) \cap models(\alpha)) \cup (2^{prop} - models(\alpha)) \subseteq models(\beta) \cup (2^{prop} - models(\alpha))\rightarrow \\
    &(models(\Sigma) \subseteq (2^{prop} - models(\alpha)) \cup models(\beta))\\
\end{align*}
$\longleftarrow$:\\
We then prove that if $(models(\Sigma) \subseteq (2^{prop} - models(\alpha)) \cup models(\beta))$, then it must always be true that $(models(\Sigma) \cap models(\alpha)) \cup (2^{prop} - models(\alpha)) \subseteq models(\beta) \cup (2^{prop} - models(\alpha)$.\\
Since $models(\Sigma) \cap models(\alpha) \subseteq models(\Sigma)$ and $models(\Sigma) \subseteq (2^{prop} - models(\alpha)) \cup models(\beta)$, it must be true that $models(\Sigma) \cap models(\alpha) \subseteq (2^{prop} - models(\alpha)) \cup models(\beta)$. Then the 2 set at left hand side are both contained by set at right hand side, and since both sides only has set union, the original statement must be true. \\[10pt]
Therefor, we have that 
\begin{align*}
    &(models(\Sigma) \cap models(\alpha)) \cup (2^{prop} - models(\alpha)) \subseteq models(\beta) \cup (2^{prop} - models(\alpha))\Leftrightarrow \\
    &(models(\Sigma) \subseteq (2^{prop} - models(\alpha)) \cup models(\beta))\\
\end{align*}
Then follow the incomplete proof above:
\begin{align*}
    & (models(\Sigma) \cap models(\alpha)) \cup (2^{prop} - models(\alpha)) \subseteq models(\beta) \cup (2^{prop} - models(\alpha))\\
    & \text{Iff  } models(\Sigma) \subseteq (2^{prop} - models(\alpha)) \cup models(\beta)\\
    & \text{Iff  } models(\Sigma) \subseteq models(\alpha \rightarrow \beta)\\
    & \text{Iff  } \sigma \models (\alpha \rightarrow \beta)
\end{align*}
Thus proved. 
\subsection*{Problem 9}
$\bullet$ If either $\sigma \models \alpha$ or $\sigma \models \beta$, then: 
\begin{align*}
    &models(\Sigma) \subseteq models (\alpha) \text{ or } models(\Sigma) \subseteq models(\beta)\\
    &models(\Sigma) \subseteq models(\alpha) \cup models(\beta)\\
    &models(\Sigma) \subseteq models(\alpha \vee \beta)\\
\end{align*}
So the first statement is true.\\[5pt]
$\bullet$ Let Prop = \{p, q\}. We define $\alpha = p\vee (\neg q)$, $\beta = (p \rightarrow q)$, $\Sigma = p$, we then have: 
\begin{align*}
    models(\alpha) = \{\{p\}, \{q\}\}\\
    models(\beta) = \{\{q\}, \{p,q\}, \emptyset\}\\
    models(\alpha \vee \beta) = \{\emptyset, \{p\}, \{q\},\{p,q\}\}\\
    models(\Sigma) = \{\{p\}, \{p, q\}\}\\
\end{align*}
Then it is clear that $models(\Sigma) \subseteq models(\alpha \vee \beta$, but $models(\Sigma) \nsubseteq models(\alpha)$ and $models(\Sigma) \nsubseteq models(\beta)$. So the second statement if false. 
\subsection*{Problem 10}
$\bullet$ For finite set, we have:\\[5pt]
Base Case: $\Sigma = \emptyset$, which must be independent since there's no formula in it.\\[5pt]
Induction: 
If $\Sigma$ itself is independent, then it is a independent subset of itself.\\[3pt]
Otherwise by definition, there must exist at least one $\alpha$ such that $models(\Sigma - \alpha) \subseteq models(\alpha)$.Then we have the following equation: 
\begin{equation*}
    models(\Sigma) models((\Sigma - \alpha) \cup \alpha) = models(\Sigma - \alpha) \cap models(\alpha) = models(\Sigma - \alpha)
\end{equation*}
Therefore $\Sigma$ and $\Sigma - \alpha$ are equivalent, and the later is a proper subset of the former.Since there exist a independent subset, $\Sigma'$, that is logically equivalent to $\Sigma - \alpha$ and both of subset and logical equivalence are transitive relation, $\Sigma'$ must also be a logically equivalent subset of $\Sigma$.\\[10pt]
$\bullet$ For infinite set, we can construct one as follow:\\[5pt]
Let the infinite set be $\Theta = (\theta_1, \theta_2, ....)$, where we recursively define $\theta_1 = p_1$, $\theta_{n+1} = (\theta_{n} \vee p_{n+1})$.\\[3pt]
Here, each $\theta_{n+1}$ is implied by $\theta_n$ so the only independent subset is the set contains only a single element and the empty set. However, any singleton is not logically equivalent to $\Theta$ since for any $\theta_{n}$ it does not imply $\theta_m$ for $m>n$, and empty set cannot imply any element of $\Theta$. Therefore, such infinite set does not have any independent logically equivalent subset. 

\newpage

\section*{Substitution}
\subsection*{Statement 1:}\\
\textbf{Proof:} \\
$\forall \tau \in 2^{prop}, \phi, \theta \in {FORM}, p \in {PROP}$, we can construct a truth assignments $\sigma \in 2^{prop}$, where if $\tau(\theta) = 1, \, \sigma = \tau \cup \{p\} , \, else \, \sigma = \tau \setminus \{p\}, s.t. \, \tau(\phi[p \rightarrow \theta]) = \sigma(\phi)$. \\
\textbf{Base:} \\
Case 1: $\phi = p$, then 
\begin{itemize}
    \item if $\tau(\theta) = 1, \, \sigma = \tau \cup \{p\} , \tau(p[p \rightarrow \theta]) = \tau(\theta) = 1 = \sigma(p) = \sigma(\phi)$.
    \item if $\tau(\theta) = 0, \, \sigma = \tau \setminus \{p\} , \tau(p[p \rightarrow \theta]) = \tau(\theta) = 0 = \sigma(p) = \sigma(\phi)$.
\end{itemize}
Case 2: $\phi = q$ and $p \neq q$, then $\tau(q[p \rightarrow \theta]) = \tau(q) = \sigma(q) = \sigma(\phi)$.\\
\textbf{Inductive:} \\
Case 3: $\phi = (\neg \phi_{0})$. Then 
\begin{align*}
    &\tau(p[p \rightarrow \theta])  \\
    &= \tau((\neg \phi_{0})[p \rightarrow \theta])\\
    &= \tau(\neg \phi_{0}[p \rightarrow \theta]) \\
    &= \neg (\tau(\phi_{0}[p \rightarrow \theta]))\\
    &= \neg(\sigma(\phi_{0}))\\
    &= \sigma (\neg \phi_{0})\\
    &= \sigma (\phi)
\end{align*}
Case 4: $\phi = (\gamma \circ \psi)$. Then 
\begin{align*}
    &\tau(p[p \rightarrow \theta])  \\
    & = \tau((\gamma \circ \psi)[p \rightarrow \theta]) \\
    & = \tau(\gamma[p \rightarrow \theta] \circ \psi[p \rightarrow \theta])\\
    & = \tau(\gamma[p \rightarrow \theta]) \circ \tau(\psi[p \rightarrow \theta])\\
    & = \sigma(\gamma) \circ \sigma (\psi)\\
    & = \sigma(\gamma \circ \psi) \\
    & = \sigma(\phi) \\
\end{align*}
Since $\phi$ is valid, then $\forall \sigma \in 2 ^ {PROP}, \sigma(\phi) = 1$. On the other hand, $\forall \tau \in 2 ^ {PROP}, \exists \tau \in 2 ^ {PROP}, \phi[p \rightarrow \theta] = \sigma(\phi) = 1$,  $\phi[p \rightarrow \theta]$ is also valid.
\subsection*{Statement 2}
$models(\alpha) = models(\beta)$  \\
\textbf{Proof:} \\
If $\alpha$ and $\beta$ are logically equivalent, then
$\phi[p \mapsto \alpha]$ and
$\phi[p \mapsto \beta]$ are logically equivalent. \\
$models(\phi[p \mapsto \alpha]) = models(\phi[p \mapsto \beta])$. \\
\textbf{Base:}
\begin{itemize}
    \item $\phi = p, models(p[p \mapsto \alpha]) = models(\alpha) = models(\beta) = models(p[p \mapsto \beta]). $
    \item $\phi = q, models(q[p \mapsto \alpha]) = models(q) = models(q[p \mapsto \beta]). $
\end{itemize}
\textbf{Inductive:}\\
\begin{itemize}
    \item
    $\phi = (\neg\psi), models(\phi[p \mapsto \alpha]) = models(\neg \psi[p \mapsto \alpha]) = 2^{PROP} \setminus models(\phi[p \mapsto \alpha]) = 2^{PROP} \setminus models(\phi[p \mapsto \beta]) = models(\neg \psi[p \mapsto \beta]) = models(\phi[p \mapsto \beta])$
    \item
    $\phi = (\psi_{0} \circ \psi_{1}), models(\phi[p \mapsto \alpha])= models(\psi_{0} [p \mapsto \alpha] \circ \psi_{1} [p \mapsto \alpha]) = models((\psi_{0} [p \mapsto \alpha]) \cup models(\psi_{0} [p \mapsto \alpha])) = models((\psi_{0} [p \mapsto \beta]) \cup models(\psi_{0} [p \mapsto \beta])) = models(\psi_{0} [p \mapsto \beta] \circ \psi_{1} [p \mapsto \beta]) = models(\phi[p \mapsto \beta])$.
\end{itemize}


\newpage
\section*{Puzzle}
%Write your solution to problem 5 here. 
Let the statement $i^{th}$ box have the gold be $b_i$. Let the statement $i^{th}$ clue is true be $c_i$.\\[5pt]
Since only one box have the gold, we have $((b_1 \vee b_2 \vee b_3) \wedge ( \neg (b_1 \wedge b_2) \wedge \neg (b_2 \wedge b_3) \wedge \neg (\b_3 \wedge b_1)))$.\\
Since only one clue is true we have $((c_1 \vee c_2 \vee c_3) \wedge ( \neg (c_1 \wedge c_2) \wedge \neg (c_2 \wedge c_3) \wedge \neg (c_3 \wedge c_1)))$.\\
And according to the clue, we have: $c_1 \Leftrightarrow \neg b_1$, $c_2 \Leftrightarrow \neg b_2$, and $c_3 \Leftrightarrow b_2$.\\[5pt]
To solve the puzzle, we need to find a truth assignment that makes all statement stated above true. The only satisfying assignment is the following:
\begin{equation*}
    b_1 = 1, b_2 = 0, b_3 = 0, c_1 = 0, c_2 = 1, c_3 = 0
\end{equation*}
According to the statement definition, the first box contains the gold.


\end{document}
