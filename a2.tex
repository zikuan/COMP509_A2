\documentstyle[11pt,abstract]{article}
\headheight 0pt
\setlength{\textwidth}{6.25in}          
\setlength{\oddsidemargin}{0in}   
\setlength{\evensidemargin}{.25in}  
\input math
\newcommand{\rarrow}{\rightarrow}
\newcommand{\darrow}{\leftrightarrow}
\newcommand{\logequiv}{{\models =\!\!\!|}}
\renewcommand{\phi}{\varphi}
\begin{document}
\pagestyle{empty}

\begin{center}
{\Large COMP 409\\
Assignment No. 2}\\
\bigskip
{\bf Due date: In class - Sept.~26, 2019}
\end{center}

\noindent
{\bf Note}:
\begin{enumerate}
\item
The submitted assignment needs to be typeset in LaTeX.
\item
{\bf Teamwork}: You are expected to complete this assignment in pairs. 
By signing your name on the assignment you are asserting
that the work submitted was done collaboratively, and you have
adhered to Rice University's Honor Code.
At the very least, this entails:
\begin{itemize}
\item
discussing each problem and agreeing on a sketch of the solution,
\item
reviewing the written solutions for technical correctness and
typographical errors,
\item
doing a joint codewalk over all written programs,
\item
dividing labor equally (roughly), and
\item
being able to explain every answer as if it is your own answer.
\end{itemize}
\item
Note on the submission the total number of hours put in by
each partner separately and by the two partners together.
\end{enumerate}

\bigskip
\noindent
{\sf Propositional Semantics}:
\begin{enumerate}
%\item
%Let $\varphi$ be a formula and let $\tau$ be a truth assignment.
%Show that $\tau \models \varphi$ iff $\varphi(\tau)=1$ iff
%$\tau\in\mbox{models}(\varphi)$.
\item
Implement a {\em truth evaluator\/} $\mbox{\sf eval}(\phi,\tau)$, 
which evaluates whether a formula $\phi$ holds for a truth assignment
$\tau$. Use the evaluator to answer the following question:
\begin{quote}
Let $\alpha$ be the formula:
\[
\begin{array}{l}
((p_1 \rightarrow (p_2 \wedge p_3)) \wedge 
((\neg p_1) \rightarrow (p_3 \wedge p_4))).
\end{array}
\]
Let $\beta$ be the formula:
\[
\begin{array}{l}
((p_3 \rightarrow (\neg p_6)) \wedge 
((\neg p_3) \rightarrow (p_4 \rightarrow p_1))).
\end{array}
\]
Let $\gamma$ be the formula:
\[
\begin{array}{l}
((\neg (p_2 \wedge p_5)) \wedge (p_2 \rightarrow p_5))
\end{array}
\]
Let $\delta$ be the formula:
\[
\begin{array}{l}
(\neg (p_3 \rightarrow p_6))
\end{array}
\]
Evaluate the formula $\psi$
\[
\begin{array}{l}
((\alpha \wedge (\beta \wedge \gamma)) \rightarrow \delta)
\end{array}
\]
under the truth assignment $I_1$, 
where $I_1(p_1)=I_1(p_3)=I_1(p_5)=0$ 
and $I_1(p_2)=I_1(p_4)=I_1(p_6)=1$, as well as
under the truth assignment $I_2$, 
where $I_2(p_1)=I_2(p_3)=I_2(p_5)=1$ 
and $I_2(p_2)=I_2(p_4)=I_2(p_6)=0$.
\end{quote}
\end{enumerate}

\noindent
{\sf Validity}: 
\begin{enumerate}
\item
Prove the validity of the following principles:
\begin{enumerate}
\item
$(((p\rightarrow q) \rightarrow p)\rightarrow p)$  - Peirce's  Law
\item
$(false \rightarrow p)$ - Ex Falso Quolibet (Here $false$ is a
formula that is always false.)
\item
$((\neg p) \vee p$) : Excluded Middle (``Tertium Non Datur'')
\item
$((p\rightarrow q) \vee p)$: Weak Excluded Middle
\end{enumerate}
\item
Prove \emph{Disjunctional Propositional Completeness}:
if $\models (\alpha\vee\beta)$ and $AP(\alpha)\cap AP(\beta)=\emptyset$,
then $\models\alpha$ or $\models\beta$
\end{enumerate}

\noindent
{\sf Logical Implication}:
\begin{enumerate}
\item
A binary connective $\circ$ is \emph{commutative} if
$(\theta\circ\psi)$ is logically equivalent to $(\psi\circ\theta)$;
it is \emph{associative} if
$((\varphi\circ\theta)\circ\psi)$ is logically equivalent to 
$(\varphi\circ(\theta\circ\psi))$.

Analyze the binary connectives $\wedge$, $\vee$, $\rightarrow$,
and $\leftrightarrow$ to see whether thet are commutative and
associative. Prove your claims.
\item
Prove de Morgan's Laws
\begin{itemize}
\item
$\models ((\neg(p\wedge q)) \darrow ((\neg p) \vee (\neg q))))$
\item
$\models ((\neg(p\vee q)) \darrow ((\neg p) \wedge (\neg q))))$
\end{itemize}
\item
Show that logical and material equivalence coincide, that is,
$\varphi \logequiv \psi$ iff $\models (\varphi\leftrightarrow\psi)$.
\item
Show that $\varphi\models\psi$ iff $(\varphi\wedge\neg\psi)$ is not
satisfiable.
\item
Prove that logical implication is reflexive, transitive, but not symmetric.
Prove that logical equivalence is an equivalence relation.
\item
Show that if $\phi$ and $\psi$ are logically equivalent and $\phi$ is
valid, then $\psi$ is valid.
\item
Which of the following two sentences implies the other?

$$(p \leftrightarrow (q \leftrightarrow r)),$$
$$((p\wedge (q \wedge r))\vee((\neg p)\wedge ((\neg q)\wedge
(\neg r)))).$$

\item
Show that $\Sigma,\alpha \models \beta$ 
iff $\Sigma \models (\alpha \rightarrow \beta)$ 
(Note: $\Sigma,\alpha$ is a shorthand for $\Sigma\cup\{\alpha\}$.)

\item
Prove or refute the following statements (here $\Sigma$ is a set of
formulas):
\begin{itemize}
\item
If either $\Sigma \models \alpha$ or $\Sigma \models \beta$, then
$\Sigma \models (\alpha \vee \beta)$.
\item
If $\Sigma \models (\alpha \vee \beta)$, then
either $\Sigma \models \alpha$ or $\Sigma \models \beta$.
\end{itemize}

\item
A set $\Sigma$ of formulas is {\em independent\/} if no member $\sigma$
of $\Sigma$ is implied by $\Sigma - \{\sigma\}$.
Show that a finite set of formulas always has an equivalent
independent subset, but this need not hold for an infinite set of
formulas.
\end{enumerate}

\bigskip
\noindent
{\sf Substitutions}:

The result of {\em substituting\/} a formula $\theta$
{\em for\/} a proposition $p$ {\em in\/} a formula $\phi$,
denoted $\phi[p \mapsto \theta]$, is defined as follows.
\begin{itemize}
\item
$p[p \mapsto \theta]=\theta$
\item
$q[p \mapsto \theta]=q$ for $q \ne p$.
\item
$(\neg\phi)[p \mapsto \theta]=(\neg\phi[p \mapsto \theta])$
\item
$(\phi \circ \psi)[p \mapsto \theta]=
(\phi [p \mapsto \theta] \circ \psi [p \mapsto \theta])$
(here $\circ$ denoted an arbitrary binary connective)
\end{itemize}
Prove the following (use induction, but carefully!):
\begin{enumerate}
\item
If $\phi$ is valid, then $\phi[p \mapsto \theta]$ is valid.
\item
If $\alpha$ and $\beta$ are logically equivalent, then
$\phi[p \mapsto \alpha]$ and
$\phi[p \mapsto \beta]$ are logically equivalent.
\end{enumerate}

\bigskip
\noindent
{\sf Puzzle}:

Three boxes are presented to you. One contains gold, the other two
are empty. Each box has imprinted on it a clue as to its contents;
the clues are (Box 1) ``The gold is not here'', 
(Box 2) ``The gold is not here'', and 
(Box 3) ``The gold is in Box 2''.  Only one message is true; 
the other two are false. Which box has the gold?

Formalize this puzzle and solve it using propositional logic.

\end{document}
